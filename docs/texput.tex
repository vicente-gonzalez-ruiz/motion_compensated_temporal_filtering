% Emacs, this is -*-latex-*-

\title{\href{https://github.com/vicente-gonzalez-ruiz/MCTF}{Motion Compensated Temporal Filtering}}

\maketitle

\section{Idea and notation}
%{{{

Motion Compensated Temporal Filtering (MCTF)~\cite{ohm1994three} is
basically a motion compensated random-access mode in which the last
P-type frame is a B-type frame (see Fig.~\ref{fig:MCTF}). MCTF can be
considered a DWT where the input samples are the original video images
and the output coefficients is a sequence of residue images. Some of
these frame-coeffs contain low frequency information (in the temporal
domain), and others represent high frequency temporal information. We
will use \emph{average}-frame (or simply \emph{average}) to refer to a
low-frequency frame, and \emph{residue} to refer to a high-frequency
frame. In general, the number of averages is smaller tha the name of
residues.

\begin{figure}
  \svg{graphics/MCTF}{500}
  \caption{The MCTF scheme.}
  \label{fig:MCTF}
\end{figure}

%}}}

\section{Objectives of MCTF}
%{{{

The main goals of
\href{https://en.wikipedia.org/wiki/Motion_compensation}{motion
  compensation} are:
\begin{enumerate}
\item Reduce the entropy of the residuals, and if
\item Increase the temporal scalability, compared to low-delay and
  random-access modes~\cite{vruiz__MC}. MCTF is a
  dyadic temporal multirresolution approach.
\end{enumerate}
  
%}}}

\section{MCTF uses ME}
%{{{

In order to exploit the temporal redundancy, the bidirectional
predictions used in MCTF are generated using ME (Motion
Estimation)~\cite{vruiz__ME}. The motion information (usually in the
form of motion vector fields with a given density\footnote{Number of
  motion vectors per pixel.}) must be known by both, the encoder and
the decoder, which runs the inverse MCTF. Therefore, MCTF can be
considered an adaptive transform, and, as many others adaptive
systems, the side information must be transmitted to or regenerated by
the decoder.

%}}}

\section{(Expected) Entropies and  dynamic ranges}
%{{{

Usually, MCTF (as many other transforms), increases the number of bits
necessary to represent the residues (coefficients), but also decreases
the entropy, because most of the information will be concentrated in a
small number of residues~\cite{vruiz__MC}.

%}}}

\section{References}
%{{{

\renewcommand{\addcontentsline}[3]{}% Remove functionality of \addcontentsline
\bibliography{image_pyramids,DWT,motion_estimation,HEVC,SVC}

%}}}
